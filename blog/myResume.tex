% Created 2017-03-18 Sat 01:23
\documentclass[11pt]{article}
\usepackage[utf8]{inputenc}
\usepackage[T1]{fontenc}
\usepackage{fixltx2e}
\usepackage{graphicx}
\usepackage{grffile}
\usepackage{longtable}
\usepackage{wrapfig}
\usepackage{rotating}
\usepackage[normalem]{ulem}
\usepackage{amsmath}
\usepackage{textcomp}
\usepackage{amssymb}
\usepackage{capt-of}
\usepackage{hyperref}
\author{彭甘雨}
\date{2017-03-17}
\title{个人简历}
\hypersetup{
 pdfauthor={彭甘雨},
 pdftitle={个人简历},
 pdfkeywords={},
 pdfsubject={},
 pdfcreator={Emacs 24.5.1 (Org mode 8.3.4)}, 
 pdflang={Zh-Cn}}
\begin{document}

\maketitle

\section*{自我介绍}
\label{sec:orgheadline1}
\begin{itemize}
\item 这份简历的作者,是一位有着70后的外貌,80后的穿着,90后的出生\ldots{}\ldots{}.的一位90后java野生程序员。
\item 14年毕业于郑州黄河科技学院,信息与计算科学专业,至今已经3年工作经验(什么?时间不对?哦,我把加班的时间也算上了);
\item 为人比较幽默,乐观,喜欢听相声,偶尔自己也会表演一下,这不,在公司年会上面表演相声还拿了一个奖;
\item 有程序员的一点修养,一切写完代码不测试就提交的行为都是耍流氓;
\item 喜欢写博客,自己申请了阿里云的机器,在里面搭了博客,但是时间很不够用,也只能偶尔瞎写一下;
\item 喜欢骑行、跑步、游戏,如果有时间的话\ldots{}\ldots{}
\item 不太喜欢用windows,当从windows转到mac的时候,我才知道有很多操作可以不用鼠标,可以节省很多时间,也不用将自己的青春献给dota;
\item 喜欢用机械键盘,你知道我手持一把Flico 87圣手茶轴黑色忍者有多么厉害么?敲起代码那就是一把梭;
\item 喜欢用emacs的org去写文字,用着实在是太爽了,导出的html方便看官在手机上面阅读。
\end{itemize}

\section*{技能介绍}
\label{sec:orgheadline2}
\begin{itemize}
\item 需求分析?不会分析需求的程序员不是好的项目经理,不写需求文档的程序员不是好的产品经理;小公司,项目紧,并且也没有人写文档,口头需求多一些,但自己还是希望规范起来;
\item 原型流程?Axure、Omnigraffle这些工具太过高大尚,没有手工画UML图、原型图来得快;如果是大的公司的话,会好一些;
\item 表结构设计? PowerDesinger太好用了,因为在mac上一直没有找到这样好用的工具;

\item 后端开发?
\item 技术亮点? 了解单点登陆,熟悉缓存配置,定时任务,SpringBoot的一些常规用法,事务模板、常用设计模式,myBaits,SpringDataJpa;
\item 前端怎么搞? 刚毕业一直在写原生js,后来用了emberJs,现在正在学reactJs;因为本人审美有问题,无论是css还是美女都不太感冒,这些东西还是交给专业的人吧;
\item 怎么部署?可以写脚本,可以用jenkins构建工具,puppet配置管理等;配置tomcat,nginx,mysql,重启服务;
\item 日常维护? 写写sql,查查日志,处理异常信息等,讨论新的需求,编写需求分析,制定开发计划,提交代码,发版\ldots{}\ldots{}
\item 业务知识:了解自动化运维的一些知识,背景,解决方案;
\item 项目规范:深刻认知项目规范的重要性,无规矩不成方圆,自己起草了公司的第一份开发规范,相比阿里公布的规范差距还是挺大的;
\item 代码管理:熟悉git代码管理,厉害了,我拉linus;
\end{itemize}

\section*{项目经验}
\label{sec:orgheadline12}
\subsection*{2013.11:入职中国软件股份有限公司--河南税务事业本部}
\label{sec:orgheadline3}
\begin{itemize}
\item 本来没有资格,但是因为老师推荐,得以通知面试。后来笔记,面试,复试,一天下来全都通过,终于得到一个实习的机会;
\end{itemize}
\subsection*{2013.11\textasciitilde{}2014.02:湖北省地税局项目}
\label{sec:orgheadline4}
\begin{itemize}
\item 自己做的第一个项目,基础不太好,恶补java、jsp、js的相关知识,最终得到开发组长的认可;
\item 主要负责:税务风险相关用例的开发
\end{itemize}
\subsection*{2014.06\textasciitilde{}2014.10:金税三期}
\label{sec:orgheadline5}
\begin{itemize}
\item 国家级项目,200人左右的开发团队
\item 项目规范、开发规范、需求文档、开发进度管理都非常规范,从中学习到了规范的重要性;
\item 主要负责:登记、优惠、申报、征收各业务模块相关用例的开发
\end{itemize}
\subsection*{2014.11\textasciitilde{}2015.05:河南省网报系统}
\label{sec:orgheadline6}
\begin{itemize}
\item 8个团队,三个月的开发时间,15年1月8号上线,后期5人运维;
\item 主要负责:企业所得税、个人所得税的申报开发、后期系统运维和增量发版
\end{itemize}
\subsection*{2015.05:入职杭州云霁科技有限公司}
\label{sec:orgheadline7}
\begin{itemize}
\item 离职原因,个人的成长已经到达瓶颈,公司框架都是封装好的,很难自己成长,每天都是写增删改查,帮助纳税人解决问题;
\item 入职原因:新兴互联网金融云计算行业,有大牛,有技术提升空间;福利好:五险一金(12\%),期权;
\end{itemize}
\subsection*{2015.06\textasciitilde{}2015.08:公司内部DevOps研发}
\label{sec:orgheadline8}
\begin{itemize}
\item 研究RPM打包原理、yum包管理原理
\item 自动将java工程文件打包成rpm包,并发布至yum源
\item 自动将rpm安装部署至目标机器
\item tar包的自动打包(已实现),安装(未实现)
\end{itemize}
\subsection*{2015.09\textasciitilde{}2015.12:公司自动化运维产品研发}
\label{sec:orgheadline9}
\begin{itemize}
\item 仿照建行的自动化运维平台做的一个工程;后来因不实用,不符合实际的业务场景;被废弃掉;
\end{itemize}
\subsection*{2016.01\textasciitilde{}2017.03:恒丰银行私有云平台项目}
\label{sec:orgheadline10}
\begin{itemize}
\item 历时6个月,15人左右的团队,上线了人员管理系统、puppet配置管理系统、单点登陆系统、权限系统、包管理系统、cmdb系统、云管理平台系统;
\item 个人主要负责:权限系统开发、包管理系统开发、后期云管理平台开发功能开发及问题修改,需求改进工作,发版,系统正常运行;
\end{itemize}
\subsection*{2017.03\textasciitilde{}至今:广发证券云平台项目(在做)}
\label{sec:orgheadline11}
\begin{itemize}
\item 技术栈:SpringBoot,SpringMVC,SpringWeb,SpringSecurity,SpringDataJpa,Mysql,Maven,Tomcat,ReactJs,代码生成工具
\item 主要负责:需求分析、表结构设计、环境搭建、登陆认证、定时任务配置、缓存配置、第三方接口对接、后端接口开发、前端开发
\end{itemize}

\section*{目前在职,准备换一份工作;离职原因如下:}
\label{sec:orgheadline13}
\begin{itemize}
\item 人受伤了,老板一直有点欺负老实人,已经996了两年了,并且经常出差,16年在恒丰出差了8个月;
\item 个人技术提升很难,因为公司小,扩张很快,项目太多,根本没有时间去学习技术,有的就是去写增删改查;并且大牛也都离职了;
\item 不喜欢传统行业,森严的制度等级,机械的工作模式,狭小的成长空间;
\item 想轻松一些,能有双休,可以让我转转杭州,来杭州快两年了,才去了两次西湖\ldots{}\ldots{}.
\item 创业公司存在着很多不规范的问题,项目规范,需求规范等,没有成熟稳定的团队,人员变动比较频繁。
\end{itemize}
\end{document}
